%----------------------------------------------------------------------------------------
%	PACKAGES AND OTHER DOCUMENT CONFIGURATIONS
%----------------------------------------------------------------------------------------

\documentclass{resume} % Use the custom resume.cls style

\usepackage[left=0.75in,top=0.6in,right=0.75in,bottom=0.6in]{geometry}
\usepackage{array}
\usepackage[]{natbib}
\usepackage{bibentry}
\usepackage{url}
\usepackage[leftmargin=5pt, parsep=11pt]{etaremune}
\usepackage[colorlinks=true,citecolor=blue,urlcolor=blue,anchorcolor = blue]{hyperref}
\usepackage{fancyhdr}
\usepackage{arydshln}
\usepackage[yyyymmdd]{datetime}
\bibliographystyle{apalike-refs}
\usepackage{fontspec}

\setmainfont{Times New Roman}
%\setmainfont{[KGMissKindergarten.ttf]}
%\setmainfont{APHont-Regular_q15c.ttf}[
%BoldFont = APHont-Bold_q15c.ttf,
%ItalicFont = APHont-Italic_q15c.ttf,
%BoldItalicFont = APHont-BoldItalic_q15c.ttf]

\newdateformat{monthdayyeardate}{%
  \monthname[\THEMONTH]~\THEDAY, \THEYEAR}
  
\newcommand{\tab}[1]{\hspace{.2667\textwidth}\rlap{#1}}
\newcommand{\itab}[1]{\hspace{0em}\rlap{#1}}
\renewcommand{\labelenumi}{[\theenumi]}
\fancyhf{}
\fancyhead{}
\fancyfoot[L]{Last updated: \monthdayyeardate\today}
\fancyfoot[R]{p. \thepage}
\pagestyle{fancy}
\renewcommand{\headrulewidth}{0pt}
\renewcommand{\dateseparator}{--}

\name{Michael Ryan Hunsaker, Ph.D.} % Your name
\position{Teacher for Students with Visual Impairments} % Your position
\address{1150 South Westside Drive, Layton, UT 84041 \\ ryhunsaker@dsdmail.net} % Your address
%\address{1001 KMBL, Brigham Young University, Provo, UT 84602 \\ naomi.hunsaker@byu.edu} % Optional address

\begin{document}
%----------------------------------------------------------------------------------------
%	Objective \& Teaching Competencies
%----------------------------------------------------------------------------------------

\begin{rSection}{My pathway from Translational Neuroscience to Special Education}
{\bf FORMATIVE EXPERIENCE}\\
It is actually rather easy to explain why I went into science and ended up as a special education teacher and advocate for disability rights and mental health. It is the result of the culminated experiences I have had with my late autistic twin brother. My professional directions and life decisions have been largely shaped by brother’s and my shared experiences, and most particularly by his death.

My path toward becoming a special educator and disabilities advocate started when I was a child. My twin brother was nonverbal and autistic. More than that, he was my best friend. I distinctly remember being visited in our home by researchers from the UCLA-University of Utah Epidemiologic Survey of Autism. Having early experiences with researchers playing brain games with me, letting me watch and ask annoying questions as the phlebotomist collect blood samples, and being able to read scientific papers in which I was a research subject was nothing short of spectacular. By the ripe old age of 10, I had made it through the two banker’s boxes of research papers my mother kept in the basement about autism. To be sure I did not understand it fully, but I was hooked. It was also about this time I decided wanted to become a neurobiologist; primarily because that was the term that the team leader of this research (Dr. Edward Ritvo) used to describe himself in one of his editorials in which he proposed studying the neurobiology of autism rather than as a mental health or purely psychological disorder.

{\bf PURSUIT OF SCIENCE \& EVOLVING PERSPECTIVES}\\
At the same time as I was learning about autism from reading these papers, I was spending a lot of time immersed within the autism community in Utah. My brother was attending what was then called the Children’s Behavioral Therapy Unit (CBTU; Now the Carmen B. Pingree Center for Children with Autism) in Salt Lake City and my mother was working there. I had the opportunity to go in at times for peer and sibling training so I could learn how to help my family manage my brother’s behavioral outbursts as well as so they could learn about how autism in the family affected the nonautistic siblings. My mother also took charge of numerous summer programs to help preteens and teens in the local autism community. This generally meant that I got to play with autistic kids and teens in the summer. In doing all of this, I spent a lot of time around autistic kids and their families. As a result, I have feel an odd sort of kinship with autistics. I have never perceived these kids as broken, but rather as just different. Some of these autistic friends did have disabling behavioral problems and epilepsy, but these seemed to be trivial when weighed against their potential, which was most often not recognized by the broader community and even more rarely achieved.

After studying biology at the University of Utah, I pursued doctoral training in neuroscience at the University of California, Davis studying neurodevelopmental and neurodegenerative disorders. My goal was to bridge the clinical research and animal model research by working in both clinical and animal labs modeling behavioral phenotypes of neurodevelopmental disorders. At first, I loved working with a dynamic team of researchers working collaboratively at different levels to understand the same disorder. 

Over time, however, I began to take notice that my underlying goal for entering science, that of helping disabled children access and succeed in education and life, was rarely shared by my colleagues. There was an omnipresent impetus to cure these populations with medication and experimental therapies. I did not enter science to cure anything; and I felt it was both naïve and intellectually dishonest to talk and act like we could.
\end{rSection}

\begin{rSection}{FORKING PATHS}
Two months before I competed my Ph.D., my brother fell ill and died unexpectedly. After his death, my frustrations with how science was approaching autism intensified. I could no longer overlook clinicians and researchers treating autistics like my brother as broken. I decided that, as a society, we are unwilling to teach disabled kids the skills they need to be successful. The lesson I took away from my graduate work was that I needed to work just as hard to protect children from the culture of doctors and researchers that were trying to “help” disabled kids to appear normal rather than focusing on developing assistive technologies and academic supports to help them achieve their potential.

After I received my Ph.D. I did a few years of postdoctoral work to gain experience working with children with Down syndrome and autism. I then made the decision to leave academic science and pursue a career in special education. I felt an intense need to apply the behavioral analytic skills that I have developed over my scientific career to provide direct services to autistic children and those with other disabilities or mental health challenges that impede their educational achievement.

In both academic research and special education contexts, I have experience working directly with children and adolescents with the following diagnoses: ADHD (all subtypes), Angelman syndrome, autism, borderline personality disorder, bipolar disorder, corpus callosum agenesis, cerebral palsy (primarily spastic quadriplegia and spastic tetraplegia), Down syndrome, 46,XXYY syndrome (one form of Kleinfelter’s), fragile X-associated disorders such as fragile X syndrome and the fragile X premutation, obsessive-compulsive disorder, septo-optic dysplasia, schizophrenia, Timothy syndrome, Tourette syndrome, traumatic brain injury, Turner syndrome, 22q11.2 (deletion and duplication syndrome), and William syndrome. More recently I have expanded my educational outreach efforts to include individuals with sensory impairments, particularly visual impairments/blindness, including individuals with optic nerve hypoplasia/septo-optic dysplasia. I also have experience in special educational contexts providing behavioral and academic intervention services for students with dual diagnosis (cognitive or developmental disabilities with concomitant mental/behavioral health challenges).
\end{rSection}
\begin{rSection}{PATH 1: SCIENTIFIC LEGACY\\DISABILITY \& MENTAL HEALTH ADVOCACY}

{\bf IDENTIFY CHALLENGES}\\
Concomitant developmental or cognitive disability with mental/behavioral health needs often prevents access to disability and mental/behavioral health services. Within the US educational system, we determine services based on a primary disability rather than programming for combinations and interactions among disabilities.

With my background in neuroscience research and special education teaching in the United States, I bring a unique perspective to disabilities advocacy. I use my experience to unpack the unique contributions of developmental and mental/behavioral health disabilities, as well as the combination of the two, in the individual educational and behavioral challenges children face.

{\bf COLLECT DATA}\\
I characterize the strengths and weaknesses of at-risk students to create their unique needs profiles that can be used to guide school teams and families in designing special education services and delivery. I also apply knowledge regarding the neurological consequences of any developmental disorders or mental health issues. In research, I referred to the data resulting from this type of an approach as a “neurocognitive endophenotype”.

I apply this process in special education settings by generating a comprehensive psychoeducational endophenotype. To do this, I collect all present and historical data regarding the child from their psychological, educational, medical, mental health, and/or clinical evaluations. I transform these data into coherent and easily interpreted forms that can be used by school teams to guide decision making.

{\bf EVIDENCE-BASED INTERVENTION}\\
Once the family and school team decide upon the interventions most likely to help the child succeed, I create an unbiased, independent evaluation of the intervention options being considered.

I bring a unique perspective to evaluating educational intervention programs. My experience in writing and reviewing scientific literature has prepared me to evaluate the evidence from the scientific literature against the claims made by the publisher.

My ability to apply scientific experience to evaluate the quality of the evidence base for interventions is becoming increasingly important every year. In the past 5 years, there has been a precipitous increase in companies providing behavioral or cognitive interventions that rely on verbiage and data from neuroscience as the primary evidence base. Unfortunately, these data and ideas from academic research are not always as clear cut as the claims made based on them.
\end{rSection}
\begin{rSection}{PATH 2: DIRECT SERVICE\\STUDENTS WITH DISABILITIES \& SENSORY IMPAIRMENTS}
I pursued doctoral training in neuroscience at UC Davis studying disorders associated with fragile X syndrome, 22q11.2 deletion syndrome (DiGeorge/velocardiofacial syndrome), autism, traumatic brain injury, and severe anxiety disorders. I was attempting to bridge the clinical research and animal model research by working in both clinical and animal labs modeling behavioral phenotypes of neurodevelopmental disorders. I learned to work as a team, as I was part of a research consortium and was working directly with five professors. I also learned how to communicate with children that had disabilities as well as their families. I also learned just how far we have to go to truly help these individuals from academia. 

After I received my Ph.D. I did a few years of postdoctoral work to gain experience working with children with Down’s Syndrome and autism. Then I left academics. I pursued teaching as a course as I felt an intense need to take the behavioral skills that I had developed over my scientific career and provide direct services to children on the autism spectrum and those with other neurodevelopmental disorders. I pursued work as a paraprofessional for a year in Alpine School District in a small group autism/life skills classroom as well as a 504 paraprofessional to help a student on the autism spectrum focus in class and complete assignments. I then was able to work as a teacher of record in a small group autism/life skills classroom with 8-12, 2-5th-grade students. Primarily I was tasked with improving student behavior as only two of the students in the classroom qualified for alternative testing (DLM) under district guidelines. 

Next, I moved over to Granite School District and worked half-time as a mainstream specialist (mainstreaming students with the goal of moving them out of self-contained units and into general education with resource services) and half-time as a resource teacher at another school. To meet the challenges of these positions I developed methodological pipelines and designed computational resources to assist in mainstreaming. The methods and results of my mainstreaming methods have been published in a scholarly journal. I also had the opportunity to work closely with the state UCAT team and within the district for implementing AAC and other assistive technology resources for students that were only in special classes because these technologies had never been implemented. 

The subsequent year I moved into the district office and worked as a mainstreaming/inclusion specialist and coordinator for Applied Academic classrooms throughout the district for elementary aged students. I also was an active member of the behavioral team and autism team, which involved designing and implementing behavior plans and doing FBA/BIP as well as actual functional analyses to help these students on the spectrum control their behaviors to meet their needs in socially appropriate ways. I also started working increasingly with students that had visual impairments in this position. Often I was brought in to help with behaviors, only to find the behaviors were predictable (and thoroughly documented by the TSVI) given the visual impairments.

\end{rSection}

\begin{rSection}{HOW THESE PATHS CONVERGE\\SPECIAL EDUCATION}

As a special educator, my focus is on how each student’s disabilities contribute to a unique pattern of academic and behavioral challenges. I use these data to develop targeted academic and behavioral interventions designed to increase the individual student's capacity and thus allow them to achieve their academic potential. My teaching experience ranges from resource service delivery to self-contained settings for students with autism, specific learning disorders, multiple disabilities, and emotional disturbance classifications. These collected experience have led me to specialize in providing highly specialized instruction to students with multiple disabilities and visual impairments including optic nerve hypoplasia and students with autism and visual impairments. 

{\bf THE BRIDGE: DATA DRIVEN CURRICULUM DEVELOPMENT}\\
In bridging my advocacy and educational efforts, I develop data sheets and providing training and professional development to individuals and teams in methods used to identify which data they choose to collect to best evaluate student behavior or academic growth; how to collect these data; methods to best visualize these data, and how to evaluate these evidence to inform data-based decisions. In doing this, I have developed a suite of tools that are useful for teams to reduce and eliminate bias in their educational decisions and ensure students have access to the least restrictive educational environments.

In providing instruction to students with multiple disabilities and visual impairments/blindness, I identified a need for development of instructional and assessment tools designed specifically for the needs of these students. I am currently developing educational vision assessment tools, integrated braille and phonics curricula, and tactile discrimination techniques. To best meet the needs of this student population, all of these tools are being made available without cost to facilitate every TSVI have access to high quality pedagogy without the barrier of cost preventing its adoption. 

\end{rSection}

\vfill
\begin{center}
	Source code for this curriculum vitae (CV) is available at: \href{http://github.com/mrhunsak/Hunsaker-CV}{http://github.com/mrhunsak/Hunsaker-PS}
\end{center}
\newpage




\end{document}